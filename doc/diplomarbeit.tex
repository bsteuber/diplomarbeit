\documentclass[a4paper, bibgerm]{article}
\usepackage[utf8]{inputenc}
\usepackage[T1]{fontenc}
\usepackage{lmodern}
\usepackage{ngerman}
\usepackage{bibgerm}
\usepackage{color}
\usepackage{graphicx}
\usepackage{hyperref}

\newcommand{\defaultscale}{0.5}

\newcommand\code[1]{\texttt{#1}}

\begin{document}

\title{MagicL: Ein Arrow-basiertes Compilerbau-Framework}
\author{Benjamin Teuber}
% \date{16. Januar 2009}

% \maketitle

\begin{abstract}
  MagicL ist der Versuch eines universellen Frameworks für den Entwurf
  von Programmier- und Auszeichnungssprachen. Es soll dem
  Meta-Programmierer praktische Werkzeuge für das Parsing und Übersetzen
  neuer sowie für Generation von Code bestehender Sprachen an die Hand
  geben, wobei insbesondere (jedoch nicht ausschließlich) aus
  S-Expressions aufgebaute Sprachen unterstützt werden. Die Parser- und
  Compilererstellung erfolgt nach einem kategorientheoretisch
  motivierten Baukastenansatz, so dass beispielsweise Parser durch die
  Kombination von Arrows erzeugt werden.
\end{abstract}

% \tableofcontents

\section{Einleitung}
\label{sec:intro}

\subsection{Motivation}
\label{sec:intro:motiv}

-Modellgetriebene Softwareentwicklung sehr in Mode (Beispiele: Oslo,
?)
-Vorteile: DRY, Domänenexperten von Prog.Sprache unabhängig, ...
-Probleme: Aufwand der Metaprogrammierung

TODO: Einlesen in bestehende Frameworks

\subsection{Lisp}
\label{sec:into:lisp}

-Code-Generation und Metaprogrammierung seit fast 50 Jahren
-was können wir lernen?
--Einheitliche Syntax macht vieles leichter
--compilererweitungen durch makros

diskussion: s-exp vs. xml

diskussion: makros good or bad

\subsection{Zielsetzung}
\label{sec:intro:goal}

\section{Erster Prototyp in Ruby}
\label{sec:sexy}

\section{Arrow-basierte Implementation in Haskell}
\label{sec:magicl}

\subsection{Kategorientheoretische Grundlagen}
\label{sec:magicl:cats}



\end{document}
