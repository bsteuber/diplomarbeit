\documentclass[a4paper, bibgerm]{article}
\usepackage[utf8]{inputenc}
\usepackage[T1]{fontenc}
\usepackage{lmodern}
\usepackage{ngerman}
\usepackage{bibgerm}
\usepackage{color}
\usepackage{graphicx}
\usepackage{hyperref}

\newcommand{\defaultscale}{0.5}

\newcommand\code[1]{\texttt{#1}}

\begin{document}

\title{Ein S-Expression-basiertes Framework für die Erzeugung von Programmiersprachen}
\author{Benjamin Teuber}
\date{16. Januar 2009}

\maketitle

\begin{abstract}
  Diese Diplomarbeit stellt ein
\end{abstract}

\tableofcontents

\section{Einleitung}
\label{sec:intro}

\section{Lisp}
\label{sec:lisp}

\subsection{S-Expressions}
\label{sec:lisp:sexp}

\subsubsection{Vergleich mit XML}
\label{sec:lisp:sexp:xml}

\subsection{Makros}
\label{sec:lisp:macros}

\section{Kategorientheorie}
\label{sec:category}

\subsection{Monaden}
\label{sec:category:monads}

\subsection{Arrows}
\label{sec:category:arrows}

\section{Grundlegende Architektur}
\label{sec:basic}

\subsection{Datenmodell}
\label{sec:basic:model}

\subsection{Transformatoren}
\label{sec:basic:trans}

\subsection{Eater}
\label{sec:basic:eater}

\section{Systemkomponenten}
\label{sec:system}

\subsection{Reader}
\label{sec:system:reader}

\subsection{Pretty-Printer}
\label{sec:system:pretty}

\subsection{Makros}
\label{sec:system:macros}

\subsection{Compiler}
\label{sec:system:compiler}

\subsection{Kommandozeilen-Interface}
\label{sec:system:mage}

\section{Modellsprachen}
\label{sec:lang}

\subsection{Quellcode}
\label{sec:lang:code}

\subsection{Haskell}
\label{sec:lang:haskell}

\subsection{Compiler}
\label{sec:lang:compiler}

\subsection{Test}
\label{sec:lang:test}

\subsection{Latex}
\label{sec:lang:latex}


\section{Mögliche Erweiterungen}
\label{sec:maybe}

\subsection{Verarbeitung von Graphen}
\label{sec:maybe:graphs}

\subsection{Petri-Netze}
\label{sec:maybe:petri}

\subsection{Einheitentheorie}
\label{sec:maybe:units}

\section{Zusammenfassung und Ausblick}
\label{sec:preview}

\end{document}
