\documentclass{beamer}
\usepackage[utf8]{inputenc}
\usepackage[T1]{fontenc}
\usepackage{lmodern}
\usepackage{ngerman}
\usepackage{bibgerm}
\usepackage{color}
\usepackage{graphicx}

\mode<presentation>  {
  \usetheme{Warsaw}
  \useoutertheme{miniframes}
  \setbeamertemplate{navigation symbols}{}
}

\newcommand{\defaultscale}{0.5}
\newcommand{\pfeil}{\item[$\Rightarrow$]}

\title[MagicL]{MagicL: Ein S-Expression-basiertes Framework für
  Compiler-Konstruktion nach dem Baukastenprinzip}
\subtitle{Diplomarbeit}
\author{Benjamin Teuber}
\date{\today}
\institute{Universität Hamburg\\ 
  Fakultät für Mathematik, Informatik und Naturwissenschaften\\
  Department Informatik\\
  AOSE'08}

\begin{document}
\maketitle

\section{Einführung}
\subsection{}

\begin{frame}{Themengebiet}
  \begin{itemize}
  \item Buzzwords:
    \begin{itemize}
    \item Model-driven architecture
    \item Domain-specific languages
    \item Metaprogramming
    \end{itemize}
  \item Problem: Compiler bauen ist aufwändig
    \begin{itemize}
    \item Parser für Quellsprache
    \item Compiler in normalaler Programmiersprache
    \item Generator für Zielsprache
    \end{itemize}
  \item Wie können wir das vereinfachen?
  \pfeil Lightweight-Compiler
  \end{itemize}
\end{frame}

\section{Lisp}
\subsection{}

\begin{frame}{Exkursion: Lisp}
  \begin{itemize}
  \item Eine der ältesten Programmiersprachfamilien (LISP: 1958)
  \item Minimale, uniforme Syntax (S-Expressions $\simeq$ XML-Subset)
  \item Sprache selbst kann durch Makros erweitert werden (``The
    programmable programming language'')
  \end{itemize}

  \begin{block}{}
    ``Any sufficiently complicated C or Fortran program contains an ad hoc
    informally-specified bug-ridden slow implementation of half of
    Common Lisp.''

    - Philip Greenspun
  \end{block}
\end{frame}

\begin{frame}[fragile]{S-Expressions}
  \begin{itemize}
  \item Ein S-Expression ist entweder:
    \begin{itemize}
    \item Ein Atom, z.B. eine Zahl, ein String, eine Variable
    \item Eine Liste von S-Expressions in Notation \\ \texttt{($sexp_1$ $sexp_2$ .. $sexp_n$)}
    \end{itemize}
  \item S-Expressions werden nach dem Parsen \textit{strukturell} gespeichert.
  \item Beipiel: Funktionsdefinition in Common Lisp
  \end{itemize}
\begin{verbatim}
(defun my-add (a b)
  (+ a b))
\end{verbatim}
\end{frame}

\begin{frame}{Lisp-Makros}
  \begin{itemize}
  \item Makros sind Funktionen von Code nach Code
  \item werden bereits beim Compilieren ausgewertet
  \item Ermöglichen inkrementelle Erweiterung des Lisp-Compilers
  \item Vergleich: Präprozessor in C
    \begin{itemize}
    \item Arbeiten auf S-Expressions einfacher \& mächtiger
    \item Volle Lisp-Funktionalität steht zur Verfügung
    \pfeil Probleme wie Namenskonflikte können elegant gelöst werden
    \end{itemize}
  \end{itemize}
\end{frame}

\begin{frame}[t, fragile]{Lisp-Makros (2)}
  \begin{itemize}
  \item Beispiel: Eigenes \texttt{switch} durch \texttt{if} ausdrücken
  \end{itemize}
  \begin{block}{Macro-Eingabe}
\begin{verbatim}
(switch input
  ("Hallo"      (print "Hallo auch"))
  ("Tschüss"    (print "Bye Bye"))
  ("Wie gehts?" (print "Gut soweit")))
\end{verbatim}
  \end{block}
  \begin{block}{Erzeugter Code}
\begin{verbatim}
(if (== input "Hallo")
  (print "Hallo auch")
 elseif (== input ""Tschüss") 
  (print "Bye Bye")
 elseif (== input "Wie gehts?")
  (print "Gut soweit"))
\end{verbatim}
  \end{block}
\end{frame}

\begin{frame}{Warum Makros toll sind}
  \begin{itemize}
  \item Compiler nicht als monolithischer Block
  \item Statt dessen Bauskastenprinzip
  \item Interface für nachträgliche Erweiterungen durch den User
  \item ``Embedded DSLs'' - in die ursprüngliche Sprache integriert
  \item Alles S-Exp-basiert: Parser \& Generator fallen weg
  \item Lisp bietet praktische DSL für Makro-Definitionen 
  \end{itemize}
\end{frame}

\section{Diplomarbeit}
\subsection{Diplomarbeit}

\begin{frame}{Motivation der Diplomarbeit}
  \begin{itemize}
  \item Idee: Übernehmen des Lisp-Ansatzes für beliebige Computersprachen, z.B.
    \begin{itemize}
    \item Java
    \item VHDL
    \item HTML
    \item Latex
    \item POV-Ray
    \end{itemize}
  \pfeil Einheitliche Syntax, einheitliche IDE, einheitliche Erweiterungsmöglichkeiten
\end{itemize}
\end{frame}

\begin{frame}{Ziel der Diplomarbeit}
  \begin{itemize}
  \item Entwurf eines S-Expression-basierten Compilerbau-Frameworks
  \item Integrierte DSLs für:
    \begin{itemize}
    \item Generatoren: Compilieren S-Exp-Sprachen in Quellcode einer Backend-Sprache
    \item Compiler: Übersetzen eine S-Exp-Sprache in eine andere
    \item Parser: Lesen externen Quelltext ein und erzeugen S-Expressions
    \end{itemize}
  \item Default-Parser für S-Expression-Klammerschreibweise
  \end{itemize}
\end{frame}

\begin{frame}{Erster Prototyp}
  \begin{itemize}
  \item Ca. ein Jahr alt - proof of concept
  \item Nachbildung des Lisp-Makroprozessors
  \item In Ruby geschrieben
  \item Zwei Sexp-Sprachen:
    \begin{itemize}
    \item Ruby - kompiliert nach Ruby-Quellcode
    \item Compiler - DSL für Sprachdefinitionen, kompiliert nach Ruby-Sexp
    \end{itemize}
  \item Seit dem erfolfreichen \textit{bootstrapping} nur noch
    generierter Ruby-Code
  \end{itemize}
\end{frame}

\begin{frame}{Neue Version}
  \begin{itemize}
  \item Derzeit im entstehen
  \item In (generiertem) Haskell programmiert - der Sprache für ``theorienahe Anwendungen''
  \item Eigener Ansatz zum Compilerbau statt Makro-Nachbau
    \begin{itemize}
    \item Mathematisch fundiert: Kategorientheorie
    \item Operatoren wie in EBNF möglich
    \end{itemize}  
  \end{itemize}
\end{frame}

\begin{frame}{Grober Zeitplan}
  \begin{itemize}
  \item Ende März:
    \begin{itemize}
    \item Abschluss der Literaturrecherchen
    \item Compilieren nach Haskell
    \end{itemize}
  \item Mitte April:
    \begin{itemize}
    \item Festlegen der Compiler-Definitions-Sprache
    \end{itemize}
  \item Mitte Mai:
    \begin{itemize}
    \item Fertigstellung des Programms
    \end{itemize}
  \item Ende Juni:
    \begin{itemize}
    \item Abgabe
    \end{itemize}
  \end{itemize}
\end{frame}
\end{document}
